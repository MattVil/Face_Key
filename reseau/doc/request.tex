\subsection{Les Requêtes}
\begin{tabular}{|c|c|c|}
\hline

\multirow{}{Client -> Serveur (100-149)} & 100 & Demande la liste des IDs disponible pour un utilisateur sur un site à l'instant t \\
\cline{2-3}
& 101 & Demande le mot de passe pour un ID (accessible par l'utilisateur) sur un site \\
\cline{2-3}
& 102 & Envoie d'une photo de l'utilisateur ainsi que ses coordonnées GPS \\

\hline

\multirow{}{Serveur -> Client (200-249)} & 200 & Envoie la liste des IDs disponible pour un utilisateur pour un site à l'instant t \\
\cline{2-3}
& 201 & Envoie le mot de passe pour un ID (accessible par l'utilisateur) sur un site \\

\hline

\multirow{}{Erreurs côté serveur (400-499)} & 400 & Requête inconnue \\
\cline{2-3}
& 401 & Utilisateur introuvable \\
\cline{2-3}
& 402 & Aucun ID disponible pour l'utilisateur u sur le site web s \\

\hline
\end{tabular}

Une requête est constitué de 2 parties:
\begin{itemize}
	\item Le code de la requête (voir tableau ci-dessus)
	\item Les informations
\end{itemize}
Dans la requête, les informations sont séparées du code de la requête par un ";". Ainsi la requête est composé de la manière suivante:
\newline
\begin{tabular}{|c|c|}
\hline
CODE & INFORMATIONS \\
\hline
\end{tabular}
\newline Exemple:
\newline
\begin{tabular}{|c|c|}
\hline
100 & toto,facebook.com \\
\hline
\end{tabular}